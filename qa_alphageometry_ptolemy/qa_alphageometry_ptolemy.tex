\documentclass[11pt]{article}
\usepackage{amsmath,amssymb,amsthm}
\usepackage[margin=1in]{geometry}
\usepackage{hyperref}

\newtheorem{theorem}{Theorem}
\newtheorem{definition}[theorem]{Definition}
\theoremstyle{remark}
\newtheorem{remark}[theorem]{Remark}

\title{
A QA-Native Proof of Ptolemy's Theorem\\
\large{via Reachability and Failure Algebra}
}
\author{Will Dale}
\date{}

\begin{document}
\maketitle

%------------------------------------------------------------------------------
\begin{abstract}
We present a QA-native proof of Ptolemy's Theorem for cyclic quadrilaterals using
integer quadrance arithmetic, reachability-based time, and a deterministic failure
algebra. Unlike classical geometric proofs or heuristic solvers, the proof requires
no continuous deformation, floating-point arithmetic, or backtracking. A proof is
realized as a finite path in a QA time graph, while impossibility is certified by
explicit obstruction types. This demonstrates QA-AlphaGeometry as an ontology-level
alternative to continuous and probabilistic geometry solvers.
\end{abstract}

%------------------------------------------------------------------------------
\section{Introduction}

Ptolemy's Theorem states that for a cyclic quadrilateral with sides
$a,b,c,d$ and diagonals $p,q$,
\[
ac + bd = pq.
\]
Classical proofs rely on continuous geometry, trigonometry, or real analysis.
Modern automated solvers often depend on heuristic search or machine learning.

\textbf{Claim.}
In QA-AlphaGeometry, a geometric proof is not discovered by search or approximation;
it is \emph{reached} as a finite path in a discrete, invariant-closed state space.

%------------------------------------------------------------------------------
\section{QA Encoding of a Cyclic Quadrilateral}

\subsection{Quadrance-Based State}

All geometry is encoded using \emph{quadrances} (squared lengths).
A QA-Ptolemy state is
\[
s = (Q_{AB}, Q_{BC}, Q_{CD}, Q_{DA};\; Q_{AC}, Q_{BD};\; q\_\text{tags},\; \text{scale\_token}),
\]
where all quantities are integers.
The scale token enforces the QA Non-Reduction Axiom: no scale-collapsing
normalization is permitted.

\subsection{Cyclicity as an Integer Predicate}

Cyclicity is enforced via the exact Ptolemy discriminant condition:
\begin{equation}
\label{eq:ptolemy-disc}
\bigl(
Q_{AC}Q_{BD}
- Q_{AB}Q_{CD}
- Q_{BC}Q_{DA}
\bigr)^2
=
4\,Q_{AB}Q_{BC}Q_{CD}Q_{DA}.
\end{equation}

This replaces the geometric notion ``points lie on a circle'' with a single
integer equality. No square roots or floating-point arithmetic appear anywhere
in the legality predicate.

%------------------------------------------------------------------------------
\section{QA-AlphaGeometry v0.1 Move Set}

We use a minimal, frozen move set.

\paragraph{CONSTRUCT.}
Enumerates candidate diagonal quadrances and retains those satisfying
\eqref{eq:ptolemy-disc} and triangle quadrea nonnegativity.
Failure produces \texttt{INVARIANT\_BREAK} with reason code \texttt{PTOLEMY\_DISC}.

\paragraph{FLIP.}
Exchanges diagonal interpretation while preserving cyclic legality.

\paragraph{SCALE$_k$.}
QA-native similarity transform multiplying all quadrances by $k^2$ and updating
the scale token.

\paragraph{RELABEL.}
Vertex permutation with consistent remapping of all quadrances.

\paragraph{VERIFY\_PTOLEMY.}
Terminal check of \eqref{eq:ptolemy-disc}. Returns \texttt{SUCCESS} or a
deterministic failure type.

Every illegal move produces an explicit obstruction; no branch fails silently.

%------------------------------------------------------------------------------
\section{Worked Example: The Square}

\subsection{Initial State}

For the unit square,
\[
Q_{AB}=Q_{BC}=Q_{CD}=Q_{DA}=1,
\qquad
Q_{AC}=Q_{BD}=2.
\]

\subsection{Verification}

Compute:
\[
(2\cdot 2 - 1\cdot 1 - 1\cdot 1)^2 = 4,
\qquad
4\cdot 1\cdot 1\cdot 1\cdot 1 = 4.
\]
Thus the discriminant condition holds exactly.

\subsection{Reachability Graph}

\begin{center}
\begin{tabular}{l}
$s_0$ (sides only) \\
$\;\;\xrightarrow{\text{CONSTRUCT}}\; s_{\square}$ \\
$\;\;\xrightarrow{\text{VERIFY}}\; \texttt{SUCCESS}$
\end{tabular}
\end{center}

Alternative diagonal choices fail deterministically with
\texttt{INVARIANT\_BREAK(PTOLEMY\_DISC)} and are pruned.

\subsection{Proof Path}

\[
[\text{CONSTRUCT},\; \text{VERIFY\_PTOLEMY}]
\]

Time complexity is exactly two QA moves. The proof \emph{is} the path.

%------------------------------------------------------------------------------
\section{Failure Algebra as Proof-Theoretic Signal}

In QA-AlphaGeometry, failure is not an error state.
Each failure type is a certificate of impossibility within the fixed ontology.
For Ptolemy's Theorem, \texttt{PTOLEMY\_DISC} precisely identifies non-cyclic
configurations.

\begin{remark}
Failure modes act as causal obstructions in QA time, not as heuristics to be bypassed.
\end{remark}

%------------------------------------------------------------------------------
\section{Comparison}

Classical geometry relies on continuous deformation.
Heuristic solvers rely on search and probabilistic guidance.
QA-AlphaGeometry relies on exact reachability and algebraic obstruction.

No approximation is involved, and no backtracking is required.

%------------------------------------------------------------------------------
\section{Conclusion}

Ptolemy's Theorem emerges as a two-step reachability path in QA time.
The proof requires no continuous reasoning, no numerical approximation,
and no heuristic search.

This example demonstrates how QA replaces continuous deformation with
discrete causality and exact obstruction algebra.

\end{document}
