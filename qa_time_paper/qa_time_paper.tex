\documentclass[11pt]{article}
\usepackage{amsmath,amssymb,amsthm}
\usepackage{graphicx}
\usepackage{hyperref}
\usepackage[margin=1in]{geometry}

% Theorem environments
\newtheorem{theorem}{Theorem}
\newtheorem{lemma}[theorem]{Lemma}
\newtheorem{corollary}[theorem]{Corollary}
\theoremstyle{definition}
\newtheorem{definition}[theorem]{Definition}
\newtheorem{axiom}[theorem]{Axiom}
\theoremstyle{remark}
\newtheorem{remark}[theorem]{Remark}

\title{A Topological Phase Transition in Quantum Arithmetic Dynamics\\
\large{with QA-Native Time and Failure Algebra}}

\author{Will Dale}
\date{}

\begin{document}
\maketitle

%==============================================================================
\begin{abstract}
We prove that the state space of Quantum Arithmetic (QA), under a fixed generator set,
undergoes a \textbf{discrete topological phase transition}: the addition of a single
contraction generator collapses a fragmented reachability manifold into a single
connected component.

Unlike classical dynamical systems, QA admits \textbf{no continuous-time embedding}
compatible with legality, invariant closure, and irreversibility. Instead, time is shown
to be \textbf{intrinsic}, defined by legal reachability, bounded return depth, and phase
evolution.

We introduce a \textbf{finite failure algebra} that classifies irreversibility and show
that failure modes act as causal obstructions shaping QA time domains. All results are
validated by exact computation, not numerical approximation.
\end{abstract}

%==============================================================================
\section{Introduction}

\subsection{Motivation}

Classical dynamics presumes continuous time as a primitive.
Discrete systems typically approximate this continuous flow through timestep
discretization. Quantum Arithmetic does neither.

\textbf{Thesis:} QA reveals time as a reachability structure, not a parameter.

\subsection{Contributions}

This paper establishes:
\begin{enumerate}
    \item A \textbf{QA-native definition of time} (Axioms T0--T4)
    \item A \textbf{No Continuous Time obstruction theorem}
    \item A \textbf{topological phase transition} in QA reachability
    \item A \textbf{finite algebra of failure modes}
    \item Exact computational verification without floating-point approximation
\end{enumerate}

%==============================================================================
\section{Quantum Arithmetic State Space}

\subsection{Canonical QA States}

A QA state $s \in \mathcal{S}$ consists of a canonical seed tuple $(b,e,d,a)$ together
with its fully recomputed invariant web, including:
\begin{itemize}
    \item Role constraints: $C$ (base), $F$ (altitude)
    \item Derived invariants: $X = de$, $J = bd$, $K = da$
    \item Major axis: $2D = 2d^2$
    \item Phase tags: $\phi_9$, $\phi_{24}$
\end{itemize}

All states must satisfy the \textbf{Non-Reduction Axiom}:
no scale-collapsing normalizations are permitted.

\subsection{Generator Sets}

We define the following generators:
\begin{itemize}
    \item $\sigma$: Growth generator
    \item $\mu$: Involution (role swap)
    \item $\lambda_k$: Scaling by factor $k$
    \item $\nu$: Contraction generator
\end{itemize}

We distinguish two generator regimes:
\begin{align}
    \Sigma_0 &= \{\sigma, \mu, \lambda\} \\
    \Sigma_1 &= \{\sigma, \mu, \lambda, \nu\}
\end{align}

%==============================================================================
\section{QA-Native Time}

\begin{axiom}[T0 -- Time-as-Legal-Transition]
Let $\mathcal{S}$ be the set of QA states and $\Sigma$ a generator set.
Define the one-step legality relation:
\[
s \to_\Sigma t \quad \Longleftrightarrow \quad
\exists g \in \Sigma: t = g(s) \wedge \texttt{legal}_\Sigma(s \xrightarrow{g} t)
\]
Time in QA is the directed reachability structure induced by $\to_\Sigma$.
A single ``tick'' is one legal generator application.
\end{axiom}

\begin{axiom}[T1 -- Duration-as-Path-Length]
The QA duration between states $s,t \in \mathcal{S}$ is:
\[
\tau_\Sigma(s,t) =
\min\{k \in \mathbb{N} : s = s_0 \to_\Sigma \cdots \to_\Sigma s_k = t\}
\]
If no such path exists, $\tau_\Sigma(s,t) = \infty$.
\end{axiom}

\begin{axiom}[T2 -- Horizon / Repairability]
For horizon $k$, define:
\[
\texttt{return\_in\_k}(s,t;k) \quad \Longleftrightarrow \quad \tau_\Sigma(s,t) \le k
\]
\end{axiom}

\begin{axiom}[T3 -- No External Time Parameter]
There is no primitive $t \in \mathbb{R}$ in QA ontology.
All temporal claims must be expressible using reachability and invariants.
\end{axiom}

\begin{axiom}[T4 -- Time-Domain Invariance]
At fixed phase tag $q$,
\[
\texttt{same\_component}_q(s,t)
\Longleftrightarrow
(s \leftrightarrow_\Sigma^* t) \wedge q(s)=q(t)
\]
Components are causally isolated QA time domains.
\end{axiom}

%==============================================================================
\section{The No Continuous Time Theorem}

\begin{definition}[QA-Compatible Continuous Flow]
A map $\Phi: \mathbb{R}_{\ge 0} \times \mathcal{S} \to \mathcal{S}$ is QA-compatible if it
satisfies identity, semigroup structure, legality preservation, nontrivial continuity,
and generator consistency.
\end{definition}

\begin{theorem}[No Continuous Time]
If QA dynamics contains at least one irreversible legal edge, then no QA-compatible
continuous-time flow exists.
\end{theorem}

\begin{proof}
Assume such a flow exists.
Nontrivial continuity would require infinitesimal interpolation between discrete QA
states. This would either introduce illegal states, collapse distinct scaled embeddings
(violating Non-Reduction), or induce reversibility contradicting irreversibility.
\end{proof}

\begin{remark}
Continuous time may arise only through non-faithful observer projections
$M:\mathcal{S}\to\mathbb{R}^m$, not within QA ontology itself.
\end{remark}

\begin{corollary}
QA time computations are exact, while continuous simulations accumulate numerical error.
\end{corollary}

%==============================================================================
\section{A Topological Phase Transition in QA Dynamics}

\begin{figure}[t]
\centering
% INSERT: SCC plots
\caption{
\textbf{Topological phase transition in QA time.}
Strongly connected component counts under $\Sigma_0$ and $\Sigma_1$
for Caps$(30\times30)$ and Caps$(50\times50)$.
}
\end{figure}

\begin{theorem}[QA Reachability Phase Transition]
At fixed phase tag $q$:
\begin{itemize}
    \item Under $\Sigma_0$, QA time fragments into many SCCs
    \item Under $\Sigma_1$, all states lie in a single SCC
\end{itemize}
No intermediate generator set yields partial connectivity.
\end{theorem}

%==============================================================================
\section{Failure Algebra}

\begin{definition}[QA Failure Modes]
\[
\mathcal{F} = \{
\texttt{OUT\_OF\_BOUNDS},
\texttt{PARITY},
\texttt{FIXED\_Q},
\texttt{INVARIANT\_BREAK},
\texttt{NON\_REDUCTION}
\}
\]
\end{definition}

\begin{table}[t]
\centering
\begin{tabular}{lr}
\hline
Failure Type & Count \\
\hline
OUT\_OF\_BOUNDS & 705 \\
PARITY & 0 \\
FIXED\_Q & 0 \\
INVARIANT\_BREAK & 0 \\
NON\_REDUCTION & 0 \\
\hline
\end{tabular}
\caption{Observed failure mode frequencies under $\Sigma_0$.}
\end{table}

\begin{theorem}[Failure Algebra Collapse]
The addition of contraction $\nu$ eliminates all conditional failure modes,
collapsing all SCCs into a single QA time domain.
\end{theorem}

%==============================================================================
\section{Conclusion}

\textbf{Quantum Arithmetic does not approximate time. It exposes time's true structure.}

Time is reachability.
Irreversibility is algebraic.
Dynamics admits topological phase transitions without continuity.

%==============================================================================
\end{document}
