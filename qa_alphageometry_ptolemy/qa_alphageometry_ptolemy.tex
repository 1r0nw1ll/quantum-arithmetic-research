\documentclass[11pt]{article}
\usepackage{amsmath,amssymb,amsthm}
\usepackage[margin=1in]{geometry}
\usepackage{hyperref}
\usepackage{microtype}
\usepackage{enumitem}

\setlist{nosep}
\setlength{\parskip}{0.25em}
\setlength{\parindent}{0pt}

\newtheorem{theorem}{Theorem}
\newtheorem{definition}[theorem]{Definition}
\theoremstyle{remark}
\newtheorem{remark}[theorem]{Remark}

\title{
A QA-Native Proof of Ptolemy's Theorem\\
\large{via Reachability and Failure Algebra}
}
\author{Will Dale}
\date{}

\begin{document}
\maketitle

%------------------------------------------------------------------------------
\begin{abstract}
We present a QA-native proof of Ptolemy's Theorem using integer quadrance arithmetic, reachability-based time, and deterministic failure algebra. The proof requires no continuous deformation, floating-point arithmetic, or backtracking. Proofs are finite paths in a QA time graph, while impossibility is certified by explicit obstruction types.
\end{abstract}

%------------------------------------------------------------------------------
\section{Introduction}

Ptolemy's Theorem states that for a cyclic quadrilateral with sides
$a,b,c,d$ and diagonals $p,q$, $ac + bd = pq$.
Classical proofs rely on continuous geometry, while automated solvers use heuristic search.
In QA-AlphaGeometry, a proof is \emph{reached} as a finite path in a discrete state space.

%------------------------------------------------------------------------------
\section{QA Encoding of a Cyclic Quadrilateral}

\subsection{Quadrance-Based State}

All geometry is encoded using \emph{quadrances} (squared lengths). A QA-Ptolemy state is
\[
s = (Q_{AB}, Q_{BC}, Q_{CD}, Q_{DA};\; Q_{AC}, Q_{BD};\; q\_\text{tags},\; \text{scale\_token}),
\]
where all quantities are integers. The scale token enforces the Non-Reduction Axiom.

\subsection{Cyclicity as an Integer Predicate}

Cyclicity is enforced via the Ptolemy discriminant condition:
\begin{equation}
\label{eq:ptolemy-disc}
\bigl(Q_{AC}Q_{BD} - Q_{AB}Q_{CD} - Q_{BC}Q_{DA}\bigr)^2 = 4\,Q_{AB}Q_{BC}Q_{CD}Q_{DA}.
\end{equation}
This replaces ``points lie on a circle'' with an integer equality using no square roots or floating-point arithmetic.

%------------------------------------------------------------------------------
\section{QA-AlphaGeometry v0.1 Move Set}

We use a minimal, frozen move set.

\paragraph{CONSTRUCT.}
Enumerates candidate diagonal quadrances and retains those satisfying
\eqref{eq:ptolemy-disc} and triangle quadrea nonnegativity.
Failure produces \texttt{INVARIANT\_BREAK} with reason code \texttt{PTOLEMY\_DISC}.

\paragraph{FLIP.}
Exchanges diagonal interpretation while preserving cyclic legality.

\paragraph{SCALE$_k$.}
QA-native similarity transform multiplying all quadrances by $k^2$ and updating
the scale token.

\paragraph{RELABEL.}
Vertex permutation with consistent remapping of all quadrances.

\paragraph{VERIFY\_PTOLEMY.}
Terminal check of \eqref{eq:ptolemy-disc}. Returns \texttt{SUCCESS} or a
deterministic failure type.

Every illegal move produces an explicit obstruction; no branch fails silently.

%------------------------------------------------------------------------------
\section{Worked Example: The Square}

\subsection{Initial State and Verification}

For the unit square, $Q_{AB}=Q_{BC}=Q_{CD}=Q_{DA}=1$ and $Q_{AC}=Q_{BD}=2$.
Computing: $(2\cdot 2 - 1\cdot 1 - 1\cdot 1)^2 = 4 = 4\cdot 1\cdot 1\cdot 1\cdot 1$.
The discriminant condition holds exactly.

\subsection{Reachability Graph and Proof}

The proof path is $s_0 \xrightarrow{\text{CONSTRUCT}} s_{\square} \xrightarrow{\text{VERIFY}} \texttt{SUCCESS}$.
Alternative diagonal choices fail with \texttt{INVARIANT\_BREAK(PTOLEMY\_DISC)} and are pruned.
Time complexity: two QA moves.

%------------------------------------------------------------------------------
\section{Failure Algebra as Proof-Theoretic Signal}

In QA-AlphaGeometry, each failure type certifies impossibility. For Ptolemy's Theorem, \texttt{PTOLEMY\_DISC} identifies non-cyclic configurations. Failure modes act as causal obstructions, not heuristics.

%------------------------------------------------------------------------------
\section{Comparison}

Classical geometry uses continuous deformation, heuristic solvers use search and probabilistic guidance, while QA-AlphaGeometry uses exact reachability without approximation or backtracking.

%------------------------------------------------------------------------------
\section{Conclusion}

Ptolemy's Theorem emerges as a two-step reachability path in QA time with no continuous reasoning, approximation, or heuristic search. This demonstrates QA's replacement of continuous deformation with discrete causality.

\end{document}
