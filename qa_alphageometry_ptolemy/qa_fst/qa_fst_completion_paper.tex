\documentclass[12pt,a4paper]{article}

\usepackage[utf8]{inputenc}
\usepackage[T1]{fontenc}
\usepackage{amsmath,amssymb,amsthm}
\usepackage{geometry}
\usepackage{booktabs}
\usepackage{enumitem}
\usepackage{hyperref}
\usepackage{listings}
\usepackage{xcolor}

\geometry{margin=1in}

\lstset{
  basicstyle=\ttfamily\small,
  backgroundcolor=\color{gray!10},
  frame=single,
  breaklines=true,
  columns=fullflexible,
}

\newtheorem{definition}{Definition}
\newtheorem{proposition}{Proposition}

\title{A Quantum Arithmetic Completion Layer\\for Field Structure Theory\\[6pt]
\large Machine-Checkable Verification of Loop-Based Structural Physics}

\author{Will\\[4pt]
\small In collaboration with Don Briddell (Field Structure Theory)\\
\small Quantum Arithmetic Research Group}

\date{January 2026}

\begin{document}

\maketitle

% ============================================================================
% ABSTRACT
% ============================================================================

\begin{abstract}
Many non-standard physical frameworks propose geometric or structural
explanations for particle formation and mass hierarchies but lack formal,
machine-checkable verification mechanisms.  We present \textbf{Quantum
Arithmetic (QA)} as a \emph{completion layer}: a framework that formalizes
state spaces, generators, invariants, and failure modes without introducing
new physical assumptions.  We apply this framework to \textbf{Field Structure
Theory (FST)} and its applied formulation \textbf{Structural Physics (SP)},
due to Briddell, as a case study.  Key FST assertions---proton stability via
geometric superimposability and particle mass-ratio correspondence via loop
counting---are encoded as deterministic certificates with explicit failure
taxonomies.  The proton symmetry witness yields $\delta_{\mathrm{sym}}
\approx 0.00327$, well within the declared stability threshold of $0.01$.
The up/down quark ratio test shows agreement between loop and MeV
representations to within $3.5 \times 10^{-4}$.  Lambda decay bookkeeping
reproduces the proton loop count exactly ($2187 - 243 - 81 - 27 = 1836$),
while a $1.95\;\mathrm{MeV}$ discrepancy against the PDG proton mass is
logged as a structured warning rather than silently adjusted.  All
validation is reproducible from published artifacts via a single command.
\end{abstract}

% ============================================================================
% 1. INTRODUCTION
% ============================================================================

\section{Introduction: The Verification Gap in Structural Physics}

\subsection{Structural theories and the reproducibility problem}

A growing body of work in theoretical physics proposes that particles,
forces, and spacetime itself emerge from geometric or topological
organization rather than point-particle interactions.  Loop-based models,
field-structure approaches, and fractal hierarchies offer intuitive
structural explanations for phenomena that the Standard Model addresses
through symmetry groups and renormalization.  However, many such frameworks
remain \emph{qualitative}---their claims are stated in prose, illustrated
with diagrams, and supported by suggestive numerical coincidences, but they
are not expressed in a form that permits independent, automated
verification.

This gap is not unique to structural physics.  Across the sciences,
reproducibility depends on claims being stated precisely enough that a third
party can check them without recourse to the original author's
interpretation.  When claims involve specific numerical relationships---mass
ratios, symmetry thresholds, decay bookkeeping---the standard of
verification should be \emph{deterministic replay}: given the same inputs
and declared invariants, any validator must reach the same conclusion.

\subsection{Why verification matters more than agreement}

Disagreement about physical theories is scientifically healthy.  What is not
healthy is the inability to determine \emph{what exactly} a theory claims,
\emph{which of those claims are numerically testable}, and \emph{what
constitutes failure}.  A theory can be internally plausible yet externally
unverifiable if its claims are embedded in narrative rather than formal
structure.

We distinguish three levels of engagement with a structural theory:
\begin{enumerate}[nosep]
  \item \textbf{Acceptance}: believing the theory is physically true.
  \item \textbf{Verification}: confirming that stated claims are internally
        consistent and reproducible.
  \item \textbf{Falsification}: identifying specific, bounded conditions
        under which a claim fails.
\end{enumerate}
Quantum Arithmetic operates at levels (2) and (3).  It does not address
level (1)---physical truth is outside its scope.  Its purpose is to make
structural theories \emph{auditable}.

\subsection{Contribution of this paper}

This paper does not propose new physics.  It demonstrates how an existing
structural theory---Field Structure Theory (FST) and its applied formulation
Structural Physics (SP), due to Briddell---can be:
\begin{itemize}[nosep]
  \item \emph{formalized} via explicit state spaces, generators, invariants,
        and obstruction types;
  \item \emph{validated deterministically} via certificates with recomputable
        witnesses;
  \item \emph{audited for numeric drift} via a structured warning taxonomy
        that logs discrepancies without silently adjusting them.
\end{itemize}
The relationship is explicit: Briddell's paper presents the theory; this
paper completes it computationally.

% ============================================================================
% 2. OVERVIEW OF FST
% ============================================================================

\section{Overview of Field Structure Theory}

We provide a brief, non-competitive summary of FST/SP sufficient for the
formalization that follows.  For the full treatment, we refer the reader to
Briddell's companion manuscript \emph{The Plenum-to-Particle Loop Sequence
of Field Structure Theory}~\cite{briddell2026}.

\subsection{Core elements}

FST posits a non-local, space-filling structural potential called the
\textbf{Plenum}, composed of continuous action loops.  Particles form when
these loops organize into stable structures through entanglement, rotation,
and condensation.  Two complementary loop types operate simultaneously:
\textbf{Rspin} (counter-clockwise, anti-matter) and \textbf{Aspin}
(clockwise, real-matter).  Rspin loops remain deployed and dynamically
rotating; Aspin loops may condense to form interaction boundaries.

\textbf{Cloaking} occurs when condensed Aspin loopage fully encloses
deployed Rspin loopage such that no Rspin segment intersects the interaction
boundary.  In this state, Rspin is structurally present but unobservable.
The minimal space-defining unit is the \textbf{Structor}: a six-loop nucleus
formed from three deployed Rspin loops enclosed by three condensed Aspin
loops.

The framework maps loop counts onto the \textbf{Sierpinski Triangle Fractal
(STF)}, whose iteration hierarchy $\{3, 9, 27, 81, 243, 729, 2187\}$
produces ratios claimed to correspond to experimentally measured particle
mass ratios in electron volts.  FST asserts that mass, charge, and particle
stability arise from geometry and loop organization.

\subsection{Claims selected for completion}

We restrict our verification to the following explicit, numerically testable
claims from the source texts:
\begin{enumerate}[nosep]
  \item \textbf{Proton stability}: The proton (1836 loops) is stable because
        its hexagonal loop architecture satisfies a superimposability
        condition, quantified by a symmetry defect $\delta_{\mathrm{sym}}$
        below a declared threshold.
  \item \textbf{Quark ratio preservation}: The ratio of up to down quark
        masses measured in MeV matches the ratio of their loop counts.
  \item \textbf{Lambda decay bookkeeping}: The lambda particle (2187 loops)
        decomposes as $2187 - 243 - 81 - 27 = 1836$, recovering the proton
        loop count via exact integer arithmetic.
\end{enumerate}
Only these claims are addressed.  We make no judgment on the broader
physical program of FST/SP.

% ============================================================================
% 3. QA AS A COMPLETION FRAMEWORK
% ============================================================================

\section{Quantum Arithmetic as a Completion Framework}

\subsection{Philosophy: verification without reinterpretation}

Quantum Arithmetic treats numbers as \emph{geometric objects with intrinsic
structure} rather than abstract quantities.  Its foundational axiom is
\emph{non-reduction}: arithmetic operations preserve scale-bearing
information rather than collapsing it.  In the context of this paper, QA
serves as a \emph{verification layer}---it encodes the concepts of a target
theory (here, FST) into a formal structure that can be validated
deterministically, without reinterpreting or replacing the theory's own
ontology.

The core theorem of the QA certificate framework is:
\begin{equation}
  \mathrm{Capability} = \mathrm{Reachability}(S, G, I)
\end{equation}
where $S$ is a state space, $G$ is a generator set (permitted operations),
and $I$ is an invariant set (constraints that must hold).  A claim is
\emph{verified} if a witness path exists from an initial state to a target
state using only generators in $G$ while preserving all invariants in $I$.
A claim \emph{fails} if an explicit obstruction prevents such a path, and
the failure is classified by type.

\subsection{Formal objects}

For any target theory, QA introduces five formal objects:

\begin{definition}[Module Spine]
A \textbf{QA module spine} is a tuple
$\langle \Omega, \Sigma, \mathcal{G}, \mathcal{I}, \mathcal{F} \rangle$
where:
\begin{itemize}[nosep]
  \item $\Omega$ is the \textbf{pattern space} (configuration space of the
        target theory);
  \item $\Sigma$ is a \textbf{state packet} (typed fields describing any
        configuration);
  \item $\mathcal{G}$ is a \textbf{generator set} (permitted transformations);
  \item $\mathcal{I}$ is an \textbf{invariant set} (quantities preserved or
        bounded);
  \item $\mathcal{F}$ is an \textbf{obstruction taxonomy} (classified failure
        modes).
\end{itemize}
\end{definition}

For FST, these are instantiated as shown in Table~\ref{tab:spine}.

\begin{table}[h]
\centering
\caption{QA module spine for Field Structure Theory.}
\label{tab:spine}
\begin{tabular}{@{}lll@{}}
\toprule
\textbf{Object} & \textbf{Name} & \textbf{Description} \\
\midrule
\multicolumn{3}{@{}l}{\emph{Pattern Space}} \\
& Plenum $\Omega$ & Non-local configuration space of loop structures \\
\midrule
\multicolumn{3}{@{}l}{\emph{Generators}} \\
& $\sigma$ & Add one loop quantum: $\ell \to \ell + 1$ \\
& $\lambda_3$ & STF scale step: $\ell \to 3\ell$ \\
& $\mu$ & Chirality swap: Rspin $\leftrightarrow$ Aspin \\
& $\kappa$ & Condensation toggle: deploy $\leftrightarrow$ condense \\
& $\chi$ & Cloaking operator: modify observability \\
\midrule
\multicolumn{3}{@{}l}{\emph{Invariants}} \\
& $\ell_{\mathrm{total}}$ & Total loop count (mass proxy) \\
& $\varphi_{\mathrm{STF}}$ & STF basis decomposition coefficients \\
& $\delta_{\mathrm{sym}}$ & Symmetry defect (superimposability) \\
& $\mathrm{obs}$ & Observer-relative visibility (visible / cloaked) \\
\midrule
\multicolumn{3}{@{}l}{\emph{Obstruction Types}} \\
& \texttt{NOT\_IN\_STF\_BASIS} & Loop count not representable in STF basis \\
& \texttt{SYMMETRY\_DEFECT} & $\delta_{\mathrm{sym}}$ exceeds threshold \\
& \texttt{CLOAKED\_FROM\_OBSERVER} & Boundary intersection blocked \\
& \texttt{SOURCE\_NUMERIC\_DRIFT} & Source text internally inconsistent \\
\bottomrule
\end{tabular}
\end{table}

\subsection{Deterministic replay and auditability}

All QA certificates satisfy a \emph{deterministic replay} contract:
\begin{enumerate}[nosep]
  \item All data is serialized as canonical JSON (UTF-8, sorted keys, no
        whitespace).
  \item Hashes are computed as SHA-256 of the canonical representation.
  \item A manifest records the hash of every artifact required for replay.
  \item Re-execution of the validator on the same artifacts must produce
        identical output.
\end{enumerate}
This is enforced by draft-07 JSON schemas and a validator whose source code
is itself included in the manifest.

% ============================================================================
% 4. FORMALIZATION
% ============================================================================

\section{Formalization of FST in QA}

\subsection{The module spine as declarative contract}

The QA module spine for FST (artifact \texttt{qa\_fst\_module\_spine.json})
encodes the pattern space, state packet, generators, invariants, and
obstruction types described in the previous section.  It also records:
\begin{itemize}[nosep]
  \item \textbf{Source artifacts}: explicit references to Briddell's draft,
        paper summary, and cover letter;
  \item \textbf{External references}: the PDG~2024 proton mass
        ($938.272\;\mathrm{MeV}$), used as an independent reference for
        drift classification;
  \item \textbf{Submission context}: target journal, cover letter claims,
        and alignment with the companion manuscript.
\end{itemize}
The spine is \emph{declarative}: it states what must hold, not how to compute
it.  Validation logic resides in the validator, which is a separate artifact.

\subsection{Failure algebra as first-class structure}

A distinguishing feature of QA verification is that \emph{failure is
structured}.  Every validation produces either:
\begin{itemize}[nosep]
  \item a success witness (with computed metrics), or
  \item one or more \textbf{fail records}, each of the form:
\end{itemize}
\begin{equation}
  \{\texttt{move},\; \texttt{fail\_type},\; \texttt{invariant\_diff}\}
\end{equation}
where \texttt{move} identifies the validation step, \texttt{fail\_type}
classifies the failure, and \texttt{invariant\_diff} records the specific
numerical discrepancy.  This structure is enforced by a JSON schema
(\texttt{FAIL\_RECORD.v1.schema.json}).

The distinction between \emph{hard failures} and \emph{warnings} is
critical.  A hard failure (e.g., \texttt{SYMMETRY\_DEFECT}) means a declared
invariant is violated.  A warning (e.g., \texttt{SOURCE\_NUMERIC\_DRIFT})
means a downstream comparison to an external reference shows drift, but no
declared invariant of the source theory is broken.  The completion layer
\emph{logs} drift rather than \emph{correcting} it.

% ============================================================================
% 5. CERTIFICATE I: PROTON STABILITY
% ============================================================================

\section{Worked Certificate I: Proton Stability via $\delta_{\mathrm{sym}}$}

\subsection{Structural claim and formal witness}

FST asserts that the proton (1836 loops) is stable because its hexagonal
loop architecture is quasi-superimposable with the Plenum's tetrahedral
isometry template.  Specifically, the proton hexagon has two excess triangles
on three of its six sides, yielding a small but non-zero asymmetry.

We formalize this as a \textbf{symmetry witness}:
\begin{align}
  \text{side excess} &= [2, 2, 2, 0, 0, 0] \\
  L_1 &= \sum_i |\text{excess}_i| = 6 \\
  \delta_{\mathrm{sym}} &= \frac{L_1}{\ell_{\mathrm{total}}}
    = \frac{6}{1836} \approx 0.00327
\end{align}

The declared stability threshold is $\delta_{\mathrm{sym}} \leq 0.01$.
Since $0.00327 \leq 0.01$, the certificate result is \textbf{PASS}.

\subsection{Interpretation}

This is not a proof of physical stability.  It is verification that:
\begin{enumerate}[nosep]
  \item the declared symmetry defect is \emph{recomputable} from the stated
        witness data;
  \item the recomputed value matches the declared value to machine precision;
  \item the value satisfies the declared threshold.
\end{enumerate}
The certificate can be replayed independently by any party with access to
the published artifacts.  If the witness data or threshold were changed, the
validator would detect the discrepancy via hash mismatch.

% ============================================================================
% 6. CERTIFICATE II: HOMOMORPHISM
% ============================================================================

\section{Worked Certificate II: Loop-to-MeV Homomorphism}

\subsection{Ratio preservation (hard check)}

FST claims that the ratio of quark masses measured in MeV corresponds to
the ratio of their loop counts.  For the up and down quarks:

\begin{table}[h]
\centering
\caption{Up/down quark ratio test.}
\label{tab:ratio}
\begin{tabular}{@{}lccc@{}}
\toprule
\textbf{System} & \textbf{Up} & \textbf{Down} & \textbf{Ratio} \\
\midrule
MeV & 2.552 & 4.925 & 0.51817 \\
Loops & 378 & 729 & 0.51852 \\
\bottomrule
\end{tabular}
\end{table}

The absolute difference between ratios is $|\Delta| \approx 3.46 \times
10^{-4}$, well within the declared tolerance of $10^{-3}$.  The certificate
result for this test is \textbf{PASS}.

\subsection{Lambda decay bookkeeping}

FST decomposes the lambda particle (2187 loops) via subtraction of STF basis
elements:
\begin{equation}
  2187 - 243 - 81 - 27 = 1836
\end{equation}
This is exact integer arithmetic.  The validator recomputes the sum and
confirms identity with the declared result.  Certificate result:
\textbf{PASS}.

The corresponding MeV bookkeeping produces:
\begin{equation}
  1115.683 - 124.173 - 41.391 - 13.797 = 936.322\;\mathrm{MeV}
\end{equation}
This recomputation also matches the declared value exactly.

\subsection{Numeric drift as logged evidence}

The source texts claim that ``measuring in electron masses (MeV) and in
loops have identical outcomes for proton 1836.''  However, the MeV
bookkeeping result ($936.322\;\mathrm{MeV}$) differs from the PDG~2024
proton mass ($938.272\;\mathrm{MeV}$) by approximately $1.95\;\mathrm{MeV}$.

The QA validator classifies this as:
\begin{lstlisting}
{
  "move": "ENFORCE_HOMOMORPHISM",
  "fail_type": "SOURCE_NUMERIC_DRIFT",
  "invariant_diff": {
    "bookkeeping_mev": 936.322,
    "proton_mev_reference": 938.272,
    "drift": 1.95,
    "within_tolerance": true
  },
  "severity": "warning"
}
\end{lstlisting}

This is the central epistemological point of the completion layer.  The drift
is:
\begin{itemize}[nosep]
  \item \textbf{not ignored}: it is explicitly recorded with its magnitude;
  \item \textbf{not promoted to failure}: the source theory's own internal
        arithmetic is consistent (the bookkeeping subtraction is correct);
  \item \textbf{not silently adjusted}: neither the loop count nor the MeV
        values are modified to produce agreement;
  \item \textbf{bounded}: the drift ($1.95\;\mathrm{MeV}$) is within the
        declared tolerance ($5.0\;\mathrm{MeV}$) and is classified as
        \texttt{within\_tolerance: true}.
\end{itemize}

This approach is superior to three common alternatives:
\begin{enumerate}[nosep]
  \item \emph{Silent omission}: presenting only the loop bookkeeping and
        ignoring the MeV comparison.
  \item \emph{Post-hoc correction}: adjusting values to force agreement.
  \item \emph{Blanket rejection}: dismissing the entire ratio correspondence
        because one downstream comparison drifts.
\end{enumerate}
By logging drift as structured data, the completion layer allows any reader
to assess the discrepancy on its own terms.

% ============================================================================
% 7. REPRODUCIBILITY
% ============================================================================

\section{Reproducibility and Tooling}

\subsection{Artifact set}

The complete set of artifacts required for replay is published at a version-
controlled repository and consists of:

\begin{table}[h]
\centering
\caption{Published artifact set.}
\label{tab:artifacts}
\begin{tabular}{@{}ll@{}}
\toprule
\textbf{Artifact} & \textbf{Role} \\
\midrule
\texttt{qa\_fst\_module\_spine.json} & Module spine (declarative contract) \\
\texttt{qa\_fst\_cert\_bundle.json} & Worked certificates (witnesses + tests) \\
\texttt{qa\_fst\_submission\_packet\_spine.json} & Companion paper posture lock \\
\texttt{qa\_fst\_manifest.json} & SHA-256 manifest over all artifacts \\
\texttt{qa\_fst\_validate.py} & Deterministic validator + self-test \\
\texttt{schemas/*.schema.json} & Draft-07 JSON schemas (5 types) \\
\bottomrule
\end{tabular}
\end{table}

The manifest covers nine artifacts (three data files, five schemas, and the
validator source), each hashed independently.  The manifest itself records a
composite hash of all entries.

\subsection{One-command verification}

The entire validation can be executed with:
\begin{lstlisting}
python qa_fst_validate.py --validate
\end{lstlisting}
This produces a JSON report containing all computed metrics, hash values,
and any warnings or failures.  The validator also supports self-test mode
(eight internal checks) and manifest regeneration.

At the repository level, a single command validates the full QA certificate
stack---including the tetrad (four certificate directions), conjecture
ledger, and the FST module:
\begin{lstlisting}
python qa_meta_validator.py
\end{lstlisting}
The FST module is invoked as a subprocess, and its JSON output is checked
for structural conformance.  Any third party with access to the repository
can reproduce the result.

% ============================================================================
% 8. COMPLETION LAYER POSTURE
% ============================================================================

\section{Why This Is a Completion Layer, Not a Competing Theory}

\subsection{What QA does not do}

It is important to state explicitly what the QA completion layer does
\emph{not} claim:
\begin{itemize}[nosep]
  \item It does not propose alternative physics.  The Plenum, loops, STF
        hierarchy, Rspin/Aspin sectors, and cloaking mechanism are all
        Briddell's constructs, used here exactly as stated.
  \item It does not reinterpret FST concepts.  The generators
        ($\sigma, \lambda_3, \mu, \kappa, \chi$) are named translations of
        operations described in the source texts, not new formalisms.
  \item It does not ``correct'' the theory.  Where numeric drift exists
        (Section~6.3), it is logged, not adjusted.
\end{itemize}

\subsection{What QA adds}

The completion layer adds three capabilities that did not previously exist
for FST:
\begin{enumerate}[nosep]
  \item \textbf{Explicitness}: Claims are encoded as typed fields with
        declared thresholds, making it unambiguous what constitutes success
        or failure.
  \item \textbf{Replayability}: Any third party can recompute every metric
        from the published artifacts and verify hash integrity.
  \item \textbf{Structured failure}: When something does not match---whether
        an internal inconsistency or an external reference mismatch---the
        discrepancy is classified by type, bounded by magnitude, and
        recorded for inspection.
\end{enumerate}

This posture converts ambiguity into data.  It allows disagreement to be
\emph{precise} rather than rhetorical.  A critic can point to a specific
fail record; a supporter can point to a specific pass.  Neither needs to
argue about what the theory ``really means,'' because the formal encoding
makes the meaning explicit.

% ============================================================================
% 9. GENERALIZATION
% ============================================================================

\section{Generalization and Future Work}

The QA completion-layer approach is not specific to FST.  Any structural
theory that makes numerically testable claims can be formalized via the same
pattern:
\begin{enumerate}[nosep]
  \item Identify the theory's state space, permitted operations, and
        conserved quantities.
  \item Encode these as a QA module spine.
  \item Select explicit, bounded claims and construct certificates with
        recomputable witnesses.
  \item Classify failures by type and log drift against external references.
\end{enumerate}

For FST specifically, natural extensions include:
\begin{itemize}[nosep]
  \item Additional particle certificates (electron, neutron, pion) using the
        same $\delta_{\mathrm{sym}}$ metric.
  \item Extended decay-path bookkeeping for particles beyond the lambda.
  \item Formal cloaking certificates that test the observability obstruction
        (\texttt{CLOAKED\_FROM\_OBSERVER}) against specific
        sector/cloak-state configurations.
\end{itemize}

The broader program is community-driven verification via published
manifests: any researcher can fork the artifact set, modify a claim or
threshold, and observe whether the validator still passes.  This makes
scientific engagement with structural theories \emph{constructive} rather
than purely critical.

% ============================================================================
% 10. CONCLUSION
% ============================================================================

\section{Conclusion}

Structural theories of physics deserve rigorous verification mechanisms
commensurate with the specificity of their claims.  Quantum Arithmetic
provides a neutral, formal completion layer that encodes claims as typed
certificates, validates them deterministically, and classifies failures as
structured data.

Applied to Field Structure Theory, this approach demonstrates that:
\begin{itemize}[nosep]
  \item the proton stability claim is internally consistent
        ($\delta_{\mathrm{sym}} \approx 0.00327 \leq 0.01$);
  \item the loop-to-MeV ratio correspondence holds within declared
        tolerances ($|\Delta| \approx 3.5 \times 10^{-4} \leq 10^{-3}$);
  \item lambda decay bookkeeping is exact in loop space
        ($2187 - 243 - 81 - 27 = 1836$);
  \item and a $1.95\;\mathrm{MeV}$ drift against the PDG proton mass is
        logged as a bounded warning, not hidden.
\end{itemize}

The completion layer neither endorses nor disputes FST's physical claims.
It makes them \emph{auditable}.  We invite independent verification,
extension, and---where warranted---precise, structured disagreement.

% ============================================================================
% REFERENCES
% ============================================================================

\begin{thebibliography}{9}

\bibitem{briddell2026}
D.~Briddell,
``The Plenum-to-Particle Loop Sequence of Field Structure Theory (FST),''
manuscript submitted to \emph{Frontiers of Physics}, 2026.

\bibitem{pdg2024}
R.~L.~Workman \emph{et al.} (Particle Data Group),
``Review of Particle Physics,''
\emph{Phys.\ Rev.\ D} \textbf{110}, 030001 (2024).

\bibitem{qa2026}
W.~(1r0nw1ll),
``Quantum Arithmetic Research: Certificate Spine and Completion Layer,''
\url{https://github.com/1r0nw1ll/quantum-arithmetic-research}, 2026.

\end{thebibliography}

% ============================================================================
% APPENDIX
% ============================================================================

\appendix

\section{Validator Output}

The complete JSON output of \texttt{qa\_fst\_validate.py --validate} is
reproduced below for reference:

\begin{lstlisting}
{
  "result": "PASS_WITH_WARNINGS",
  "ok": true,
  "hashes": {
    "spine": "cf488a5d...eb67bf7f",
    "cert_bundle": "ea2a3fcd...147f2f22"
  },
  "metrics": {
    "delta_sym_recomputed": 0.0032679738562091504,
    "l1_recomputed": 6,
    "lambda_loops_recomputed": 1836,
    "lambda_mev_recomputed": 936.322,
    "proton_mev_drift": 1.95,
    "u_d_loop_ratio_recomputed": 0.5185185185185185,
    "u_d_mev_ratio_recomputed": 0.5181725888324873,
    "u_d_ratio_abs_delta": 0.0003459296860311989
  },
  "warnings": [
    {
      "move": "ENFORCE_HOMOMORPHISM",
      "fail_type": "SOURCE_NUMERIC_DRIFT",
      "severity": "warning",
      "invariant_diff": {
        "bookkeeping_mev": 936.322,
        "proton_mev_reference": 938.272,
        "drift": 1.95,
        "within_tolerance": true
      }
    }
  ]
}
\end{lstlisting}

\section{SHA-256 Manifest}

The manifest covers nine artifacts.  A representative excerpt:

\begin{lstlisting}
Module spine:      cf488a5d8309feaf...
Cert bundle:       ea2a3fcd2747b0c6...
Submission packet: e31afb0970baefdd...
Validator source:  7cae088dcd7daabf...
Manifest hash:     9a3f68f071293b9f...
\end{lstlisting}

Full hashes and all schema entries are recorded in
\texttt{qa\_fst\_manifest.json}.

\end{document}
