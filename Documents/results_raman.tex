% Raman Spectroscopy: QA‑Friendly Encodings (Final)

\subsection{Raman Spectroscopy: QA as a One‑Shot Coordinate}

We evaluated QA on real Raman spectra after lightweight, physics‑aware preprocessing (baseline removal via moving average, clamping/area normalization on a common grid, and data‑driven windows). We compared several 2D encodings \((b,e)\) and then applied QA‑21/27 invariants to the same \((b,e)\) inputs. Three encodings form the core result:

\paragraph{Fundamental–Overtone (FO v2; baseline‑corrected + dynamic windows).}
Map $b$ to the normalized fingerprint fundamental location and $e$ to the log intensity ratio between stretch and fingerprint dominants. On our 12‑class Raman set, a linear classifier improves from \textbf{0.24} (raw) to \textbf{0.55} (QA‑21), and a shallow MLP from \textbf{0.32} (raw) to \textbf{0.58}. Unsupervised ARI remains small but non‑zero (raw $\approx$ 0.17, QA‑21 $\approx$ 0.05). This mirrors the circles/moons behavior: QA makes the problem far more linear with no labels or extra features.

\paragraph{Fingerprint Centroid+Sharpness (windowed).}
Within the fingerprint band, let $b$ be the normalized energy centroid and $e$ the log width. Here QA‑21 lifts both supervised and unsupervised metrics: ARI \textbf{0.03} $\to$ \textbf{0.056}, LogReg \textbf{0.30} $\to$ \textbf{0.43}, and MLP \textbf{0.29} $\to$ \textbf{0.44}. This is a simple, robust 2D encoding where QA helps across the board.

\paragraph{Fingerprint Multi‑Segment (3 subbands).}
We split the fingerprint band into three equal‑energy subbands and form three local pairs \((b_k,e_k)\), each describing centroid and sharpness in its subband, then concatenate QA features. This yields balanced, high performance: LogReg \textbf{0.48} $\to$ \textbf{0.61}, MLP \textbf{0.59} $\to$ \textbf{0.64}, while ARI remains high (\textbf{0.265} $\to$ \textbf{0.238}).

\paragraph{Takeaway.}
Once \((b,e)\) encodes spectral physics (fundamental–overtone structure, fingerprint centroids/widths, or multi‑segment shapes), QA behaves as a universal one‑shot coordinate: linear/shallow models become strong with minimal features and without task‑specific learning. Earlier “naïve” encodings (e.g., raw peak spacings without baseline/windowing) did not show this behavior, underscoring that QA’s strength appears when the 2D map aligns with domain geometry.

\subsection{Raman as a QA Graph (Bridge to QA‑GNN)}

We also form QA graphs where nodes are Raman samples and edges come from k‑NN in QA space. On FO v2 (single tuple), baseline spectral clustering is highly modular (Q~0.86) but moderately label‑aligned. Moving to a multi‑tuple graph (FO v2 $+$ fingerprint multiseg) improves Purity and NMI (0.4368$\to$0.4753 and 0.2684$\to$0.3294, respectively), while QA‑X/full provide modest Q lifts without degrading alignment. This suggests QA’s primary benefit for Raman graphs comes from increasing node capacity via multi‑tuple encodings, with QA kernels acting as a mild fine‑tuner. See Table~\ref{tab:raman-graph} for a compact comparison.
