% =========================
% Appendix A — Unified Curvature Framework (arXiv-ready)
% Merges: (i) Gradient/CD-1 theorem, (ii) QARM sibling, (iii) Monoidal bottleneck
% Document ID: QA_CURVATURE_UNIFICATION_APPENDIX.v1 (LaTeX)
% Date: 2026-02-19
% Source: ChatGPT architectural review (2026-02-19)
% Sibling: docs/theory/QA_CURVATURE_UNIFICATION_APPENDIX.md
% =========================

\appendix
\section{Unified Curvature Framework for QA Generator Dynamics}
\label{app:qa-curvature-unified}

This appendix unifies three curvature theorems governing certified stability in Quantum Arithmetic (QA) dynamics:
(i) gradient stochastic dynamics (SGD/CD-$k$ class), (ii) discrete QARM generator dynamics, and
(iii) monoidal composition of QA dynamical systems. All three are instances of a single
\emph{curvature--contraction with dispersion} principle and map directly to Gate~3/4 certification anchors.

\subsection{A.1 Unified setup and notation}
Let $\mathcal{X}$ be a finite-dimensional real vector space with norm $\|\cdot\|$.
Let $Q:\mathcal{X}\to\mathcal{X}$ be a symmetric idempotent projection:
\[
Q^2 = Q,\qquad Q^\top = Q.
\]
Define the deviation component
\[
\Delta_t := Q x_t.
\]
We consider discrete-time dynamics of the affine-noise form
\begin{equation}
x_{t+1} = x_t - \eta_t\big(D(x_t) + \zeta_t\big),
\tag{A.1}\label{eq:A1}
\end{equation}
where $D$ is a restorative drift operator, $\zeta_t$ is a disturbance/noise term, and $\eta_t>0$ is a step size.
Projecting \eqref{eq:A1} by $Q$ yields the deviation dynamics
\begin{equation}
\Delta_{t+1} = \Delta_t - \eta_t\big(QD(x_t) + \xi_t\big),\qquad \xi_t := Q\zeta_t.
\tag{A.2}\label{eq:A2}
\end{equation}
All results below are statements about the deviation process $\{\Delta_t\}$.

\subsection{A.2 Gradient generator curvature theorem (SGD/CD-$k$ regime)}
This covers SGD, CD-1/CD-$k$, and related stochastic gradient systems, as certified in the closed-form special case
(e.g., Hessian $H=\lambda Q$).

\paragraph{Assumptions (G1--G4).}
Let $R:\mathcal{X}\to\mathbb{R}$ be a restoring potential and consider $D(x)=\nabla R(x)$.
Assume:

\begin{itemize}
\item[(G1)] (\emph{Deviation projection}) $\Delta_t=Qx_t$ with $Q^2=Q$, $Q^\top=Q$.
\item[(G2)] (\emph{Update law}) $x_{t+1}=x_t-\eta_t(\nabla R(x_t)+g_t)$, so
$\Delta_{t+1}=\Delta_t-\eta_t(Q\nabla R(x_t)+\xi_t)$ with $\xi_t:=Qg_t$.
\item[(G3)] (\emph{Restricted strong convexity on deviation modes}) There exists $\lambda>0$ such that
\[
\langle \Delta,\, Q\nabla R(x)\rangle \ge \lambda\|\Delta\|^2.
\tag{A.3}\label{eq:A3}
\]
A sufficient condition is that $R(x)=\widetilde R(Qx)$ (so $Q\nabla R(x)=Q\nabla R(Qx)$), or that
$Q\nabla^2R(x)Q\succeq \lambda Q$.
\item[(G4)] (\emph{Lipschitz smoothness on deviation modes}) There exists $L\ge \lambda$ such that
\[
\|Q\nabla R(x)-Q\nabla R(y)\| \le L\cdot \|Q(x-y)\|.
\tag{A.4}\label{eq:A4}
\]
\end{itemize}
Additionally assume the deviation noise is conditionally zero-mean with bounded second moment:
\begin{equation}
\mathbb{E}[\xi_t\mid \mathcal{F}_t]=0,\qquad
\mathbb{E}[\|\xi_t\|^2\mid \mathcal{F}_t]\le \sigma^2,
\tag{A.5}\label{eq:A5}
\end{equation}
where $\mathcal{F}_t$ is the natural filtration.

\begin{theorem}[Gradient curvature contraction]\label{thm:A1}
Under \eqref{eq:A2}--\eqref{eq:A5}, the deviation energy $V(\Delta):=\tfrac12\|\Delta\|^2$ satisfies
\[
\mathbb{E}\!\left[V(\Delta_{t+1})\mid \mathcal{F}_t\right]
\le \bigl(1-2\eta_t\lambda+\eta_t^2L^2\bigr)V(\Delta_t) + \tfrac12\,\eta_t^2\sigma^2.
\tag{A.6}\label{eq:A6}
\]
If $0<\eta_t<2\lambda/L^2$, then $\rho:=1-2\eta_t\lambda+\eta_t^2L^2\in(0,1)$ and the process is mean-square stable with
\[
\limsup_{t\to\infty}\mathbb{E}\|\Delta_t\|^2 \le \frac{\eta_t\sigma^2}{2\lambda-\eta_tL^2}.
\tag{A.7}\label{eq:A7}
\]
\end{theorem}

\paragraph{Curvature proxy.}
Define the gradient curvature margin
\[
\kappa_{\mathrm{grad}}(t) := \lambda - \frac{\eta_tL^2}{2},
\tag{A.8}\label{eq:A8}
\]
so $0<\eta_t<2\lambda/L^2$ is equivalent to $\kappa_{\mathrm{grad}}(t)>0$.

\paragraph{Closed-form certified special case (exact).}
If $\nabla^2R\equiv \lambda Q$ (equivalently: full-space Hessian $\lambda Q$, and restricted to $\mathrm{Im}(Q)$ it acts as
$\lambda I$), then the deviation linear map is $A=I-\eta_t\lambda Q$, whose deviation-subspace spectral radius is
$\rho_{\mathrm{dev}}(A)=|1-\eta_t\lambda|$, and the exact contraction margin is
\[
\widehat\kappa_{\mathrm{QA}}(t) = 1-|1-\eta_t\lambda|.
\tag{A.9}\label{eq:A9}
\]
This is the ``no-probe'' Gate~3 formula used by closed-form families.

\subsection{A.3 QARM generator curvature theorem (discrete generator algebra)}
We now replace gradient drift with discrete QA generator algebra, with explicit basin-local constants.

\paragraph{Assumptions (Q1--Q4).}
Let the projected drift be $QD(x)$ (not necessarily a gradient). Assume there exists a basin $\mathcal{B}\subset\mathcal{X}$
and constants $\lambda_b>0$, $L_b\ge 0$ such that for all times the trajectory remains in $\mathcal{B}$:

\begin{itemize}
\item[(Q1)] (\emph{Basin-local drift contraction}) For $x_t\in\mathcal{B}$,
\[
\mathbb{E}\langle \Delta_t,\, QD(x_t)\rangle \ge \lambda_b\|\Delta_t\|^2.
\tag{A.10}\label{eq:A10}
\]
\item[(Q2)] (\emph{Basin-local Lipschitz drift}) For all $x,y\in\mathcal{B}$,
\[
\|QD(x)-QD(y)\| \le L_b \|x-y\|.
\tag{A.11}\label{eq:A11}
\]
\item[(Q3)] (\emph{Generator dispersion}) The projected disturbance satisfies
\[
\mathbb{E}\|\xi_t\|^2 \le \sigma^2_{\mathrm{QARM}}.
\tag{A.12}\label{eq:A12}
\]
\item[(Q4)] (\emph{Basin containment}) $x_t\in\mathcal{B}$ for all $t$ (escape is allowed as a failure mode; see below).
\end{itemize}

\paragraph{QARM curvature.}
Define the basin-local QARM curvature margin
\[
\kappa_{\mathrm{QARM}}(t) := \lambda_b - \frac{\eta_tL_b^2}{2}.
\tag{A.13}\label{eq:A13}
\]

\begin{theorem}[Basin-local QARM stability]\label{thm:A2}
Under \eqref{eq:A2} and assumptions \eqref{eq:A10}--\eqref{eq:A12}, the deviation satisfies
\[
\mathbb{E}\|\Delta_{t+1}\|^2
\le \bigl(1-2\eta_t\lambda_b+\eta_t^2L_b^2\bigr)\|\Delta_t\|^2 + \eta_t^2\sigma^2_{\mathrm{QARM}}.
\tag{A.14}\label{eq:A14}
\]
If $0<\eta_t<2\lambda_b/L_b^2$ (equivalently $\kappa_{\mathrm{QARM}}(t)>0$), then
\[
\limsup_{t\to\infty}\mathbb{E}\|\Delta_t\|^2 \le \frac{\eta_t\sigma^2_{\mathrm{QARM}}}{2\lambda_b-\eta_tL_b^2}.
\tag{A.15}\label{eq:A15}
\]
\end{theorem}

\begin{corollary}[Escape destroys guarantees]\label{cor:A2-escape}
If basin containment (Q4) fails, then (Q1)--(Q2) are no longer guaranteed; contraction and noise-floor bounds
\eqref{eq:A14}--\eqref{eq:A15} do not apply after the escape time. In QA certification, this is a first-class obstruction
(\texttt{BASIN\_ESCAPE}) and invalidates any claim of basin-local stability for the escaped segment.
\end{corollary}

\subsection{A.4 Monoidal curvature bottleneck theorem (universal composition law)}
We now abstract both gradient and QARM forms into a universal QA dynamics object.

\paragraph{Setup.}
Let $(\mathcal{C},\otimes)$ be a monoidal category of QA dynamical systems (objects are systems; morphisms are certified maps).
Each object $X\in\mathrm{Ob}(\mathcal{C})$ carries a deviation process $\Delta_{X,t}$ and satisfies a one-step affine
contraction inequality of the form
\begin{equation}
\mathbb{E}\|\Delta_{X,t+1}\|^2 \le \alpha_X(t)\,\mathbb{E}\|\Delta_{X,t}\|^2 + \beta_X(t),
\qquad \alpha_X(t)\in[0,1),\ \beta_X(t)\ge 0.
\tag{A.16}\label{eq:A16}
\end{equation}
Define the monoidal curvature norm and noise floor proxy
\[
|\kappa|_\otimes(X,t) := 1-\alpha_X(t),\qquad
\mathrm{NF}(X,t) := \frac{\beta_X(t)}{1-\alpha_X(t)}.
\tag{A.17}\label{eq:A17}
\]

\begin{theorem}[Monoidal curvature bottleneck under $\otimes$]\label{thm:A3}
Assume parallel composition $X\otimes Y$ evolves with product deviation energy
$\|\Delta_{X\otimes Y,t}\|^2 := \|\Delta_{X,t}\|^2 + \|\Delta_{Y,t}\|^2$ and that the coupled evolution preserves the
component-wise bounds \eqref{eq:A16}. Then the composed system satisfies an affine contraction bound
\[
\mathbb{E}\|\Delta_{X\otimes Y,t+1}\|^2 \le \alpha_{X\otimes Y}(t)\,\mathbb{E}\|\Delta_{X\otimes Y,t}\|^2 + \beta_{X\otimes Y}(t),
\tag{A.18}\label{eq:A18}
\]
with bottleneck laws
\[
|\kappa|_\otimes(X\otimes Y,t) = \min\big(|\kappa|_\otimes(X,t),\,|\kappa|_\otimes(Y,t)\big),
\tag{A.19}\label{eq:A19}
\]
and
\[
\mathrm{NF}(X\otimes Y,t)\ \le\ \frac{\beta_X(t)+\beta_Y(t)}{\min\big(1-\alpha_X(t),\,1-\alpha_Y(t)\big)}.
\tag{A.20}\label{eq:A20}
\]
Consequently, a single unstable or marginal subsystem ($|\kappa|_\otimes=0$) collapses the composed curvature to zero.
\end{theorem}

\begin{corollary}[Serial composition accumulates log-contraction]\label{cor:A3-serial}
If $X$ is iterated for $T$ steps with $\alpha_X(t)\le \bar\alpha<1$ and $\beta_X(t)\le \bar\beta$, then
\[
\mathbb{E}\|\Delta_{X,T}\|^2 \le \bar\alpha^{T}\,\mathbb{E}\|\Delta_{X,0}\|^2 + \frac{\bar\beta}{1-\bar\alpha}.
\tag{A.21}\label{eq:A21}
\]
Thus log-contraction accumulates additively over time while the stationary deviation is controlled by $\bar\beta/(1-\bar\alpha)$.
\end{corollary}

\subsection{A.5 Certification mapping (Gate 3/4 anchors and obstruction algebra)}
The unified curvature objects map directly to QA certification fields.

\paragraph{A.5.1 Minimal log fields (per-family, per-epoch).}
\begin{itemize}
\item Deviation norm trace: \texttt{reg\_norm\_per\_epoch[t]} (or QARM analog).
\item Step-size trace: \texttt{lr\_per\_epoch[t]} (or QARM step-size / temperature).
\item Curvature trace: \texttt{kappa\_hat\_per\_epoch[t]}.
\item Argmin/argmax anchors: \texttt{min\_kappa\_hat}, \texttt{min\_kappa\_epoch}, \texttt{max\_dev\_norm}, \texttt{max\_dev\_epoch}.
\item Hash chain: \texttt{kappa\_hash} = SHA256(\texttt{kappa\_hat\_per\_epoch} canonical JSON).
\end{itemize}

\paragraph{A.5.2 Gate rules (canonical).}
\begin{itemize}
\item \textbf{Gate 3 (recompute)}: recompute $\widehat\kappa$ from the declared regime (closed-form or detector/probe),
then enforce exact match (within declared rounding) and verify argmin/argmax anchors.
\item \textbf{Gate 4 (hash-chain)}: verify \texttt{kappa\_hash} and any additional hash commitments (e.g., detector inputs).
\end{itemize}

\paragraph{A.5.3 Minimal obstruction list (unified).}
\begin{itemize}
\item \texttt{NEGATIVE\_GENERATOR\_CURVATURE}: recomputed $\widehat\kappa(t)<0$ for some $t$ (structural instability).
\item \texttt{CURVATURE\_RECOMPUTE\_MISMATCH}: recomputed values disagree with attested values.
\item \texttt{MAX\_DEV\_SPIKE\_ATTESTATION\_MISMATCH}: argmax(\texttt{reg\_norm}) does not match attested max fields.
\item \texttt{BASIN\_ESCAPE}: basin containment assumption (Q4) violated (QARM/nonconvex regime).
\item \texttt{DRIFT\_CONTRACTION\_VIOLATION}: empirical checks falsify (Q1)/(G3)-type contraction within the claimed basin.
\item \texttt{DISPERSION\_BOUND\_MISSING}: missing or unverifiable dispersion bound required for noise-floor claims.
\item \texttt{MOVE\_FAIL\_TYPE\_MISMATCH}: (QARM) transition log indicates failure modes inconsistent with certified regime.
\end{itemize}

\paragraph{A.6 One-line unification statement.}
The gradient theorem (Thm.~\ref{thm:A1}), the QARM sibling (Thm.~\ref{thm:A2}), and the monoidal bottleneck law
(Thm.~\ref{thm:A3}) are instances of:
\[
\textit{Stability is controlled by a curvature margin (contraction) minus dispersion (noise), and composition is bottlenecked by the weakest subsystem.}
\]
